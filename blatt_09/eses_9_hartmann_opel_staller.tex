\documentclass[a4paper]{scrartcl}


\usepackage[utf8]{inputenc}
\usepackage[ngerman]{babel}
\usepackage{enumerate}
\usepackage{tikz}
\usepackage{fancyhdr}
\usepackage{lastpage}
\usepackage{verbatim}

\usepackage{listings}
\setlength{\parindent}{0mm}
\usepackage{graphicx}
\usepackage{amsmath}
\usepackage{algorithm2e}

\pagestyle{fancy}
\fancyhead[L]{SS 2017\\Joshua Hartmann}
\fancyhead[C]{Entwurf und Synthese von Eingebetteten Systemen\\Manfred Opel}
\fancyhead[R]{Blatt 9\\Nicola Staller}

\fancyfoot[L]{}
\fancyfoot[C]{\thepage /\pageref{LastPage}}
\fancyfoot[R]{}

\renewcommand{\textheight}{700px}
\renewcommand{\footskip}{10px}
\newcommand*\xor{\mathbin{\oplus}}
\begin{document}	
\section*{Aufgabe 1: High Level Synthese: Hu- und List-Scheduling}

\begin{enumerate}[(a)]
	\item
	\item
	\item
	\item
\end{enumerate}

\section*{Aufgabe 2: High Level Synthese: Force-Directed Scheduling}

\begin{enumerate}[(a)]
	\item Zeitrahmen:
	
	\begin{tabular}{|l||c|c|}
		\hline 
		Task& $t_i^L$ & $t_i^S$ \\ 
		\hline 
		1& 1 & 3 \\ 
		\hline 
		2& 1 & 4 \\ 
		\hline 
		3& 1 & 3 \\ 
		\hline 
		4& 2 & 4 \\ 
		\hline 
		5& 2 & 4 \\ 
		\hline 
		6& 3 & 5 \\ 
		\hline 
	\end{tabular} 
	
	Operations- und Operationstypwahrscheinlichkeiten:
	
	\begin{tabular}{|l||c|c|c|c|c|c|c|c|}
		\hline 
		Zeitschritt l& $p_1(l)$ & $p_2(l)$ & $p_3(l)$ & $p_4(l)$ & $p_5(l)$ & $p_6(l)$ & $q_{ALU}$ & $q_{MUL}$ \\ 
		\hline 
		1& 1/3 & 1/4 & 1/3 & 0   & 0   & 0   & 1/3 & 7/12 \\ 
		\hline 
		1& 1/3 & 1/4 & 1/3 & 1/3 & 1/3 & 0   & 2/3 & 11/12 \\ 
		\hline 
		1& 1/3 & 1/4 & 1/3 & 1/3 & 1/3 & 1/3 & 1   & 11/12 \\ 
		\hline 
		1& 0   & 1/4 & 0   & 1/3 & 1/3 & 1/3 & 2/3 & 7/12 \\ 
		\hline 
		1& 0   & 0   & 0   & 0   & 0   & 1/3 & 1/3 & 0 \\ 
		\hline 
	\end{tabular} 
	
	\item Selbstkräfte:
	
	Berechnung:\\
	Operationstypwahrscheinlichkeit $q_k(l)$\\
	Operationswahrscheinlichkeit $p_i(l)$\\
	Selbstkraft $F_{i,l}^{self} = q_k(l) - p_i(l) \sum_{m=t_i^S}^{t_i^S}q_k(m)$
	
	
	(ausführlich für v1, aus Platzgründen für den Rest nur das Ergebnis)
	
	\begin{tabular}{|c|c|c|c|c|c|c|}
		\hline 
		Zeitschritt l& $F_{1,l}^{self}$ & $F_{2,l}^{self}$ & $F_{3,l}^{self}$ & $F_{4,l}^{self}$ & $F_{5,l}^{self}$ & $F_{6,l}^{self}$ \\ 
		\hline 
		1& $\frac{1}{3}-\frac{1}{3}\cdot(\frac{1}{3}+\frac{2}{3}+1)=-\frac{1}{3}$ & $-\frac{2}{12}$ & $-\frac{2}{9}$ & $\frac{1}{3}$ & $\frac{7}{12}$ & $\frac{1}{3}$ \\ 
		\hline 
		2& $\frac{2}{3}-\frac{1}{3}\cdot(\frac{1}{3}+\frac{2}{3}+1)=0$ & $\frac{2}{12}$ & $\frac{1}{18}$ & $-\frac{1}{9}$ & $\frac{1}{18}$ & $\frac{2}{3}$ \\ 
		\hline 
		3& $1-\frac{1}{3}\cdot(\frac{1}{3}+\frac{2}{3}+1)=\frac{1}{3}$ & $\frac{2}{12}$ & $\frac{1}{18}$ & $\frac{2}{9}$ & $\frac{1}{18}$ & $\frac{1}{3}$ \\ 
		\hline 
		4& $\frac{2}{3}-0=\frac{2}{3}$ & $-\frac{2}{12}$ & $\frac{7}{12}$ & $-\frac{1}{9}$ & $-\frac{2}{9}$ & 0 \\ 
		\hline 
		5& $\frac{1}{3}-0=\frac{1}{3}$ & 0 & 0 & $\frac{1}{3}$ & 0 & $-\frac{1}{3}$ \\ 
		\hline 
	\end{tabular} 
	\item Wir schedulen den Task mit der kleinsten Gesamtkraft. Für die Gesamtkraft addieren wir die Predecessor- und Successorkräfte auf die Selbstkräfte. In diesem Fall haben sowohl Operation 1 in Zeitschritt 1 als auch Operation 6 in Zeitschritt 5 eine minimale Gesamtkraft von $-\frac{1}{3}$.\\
	Damit würden wir Task 1 in Takt 1 starten, ebensogut könnte man auch Task 6 in Takt 5 starten
	\item Die Optimierung der Fläche unter Zeitconstraints geschieht beim kräftebasierten Schedulingansatz durch Auswahl des Tasks mit der \textbf{kleinsten} Gesamtkraft.\\
	Bei der Optimierung der Latenz unter Flächenconstraints wird der Force Directed List Schdeduling Algorithmus genutzt, bei dem die Selektionsprozedur kräftegesteuert ist. Hier wird jedoch der Task mit der \textbf{größten} Kraft ausgewählt, da wir die gegebenen Ressourcen möglichst gut auslasten wollen, um nicht unnötige Latenzen zu erhalten.
\end{enumerate}

	
\end{document}

