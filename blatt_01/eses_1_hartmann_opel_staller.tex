\documentclass[a4paper]{scrartcl}


\usepackage[utf8]{inputenc}
\usepackage[ngerman]{babel}
\usepackage{enumerate}
\usepackage{tikz}
\usepackage{fancyhdr}
\usepackage{lastpage}
\usepackage{verbatim}

\usepackage{listings}
\setlength{\parindent}{0mm}
\usepackage{graphicx}
\usepackage{amsmath}
\usepackage{algorithm2e}

\pagestyle{fancy}
\fancyhead[L]{SS 2017\\Joshua Hartmann}
\fancyhead[C]{Entwurf und Synthese von Eingebetteten Systemen\\Manfred Opel}
\fancyhead[R]{Blatt 1\\Nicolas Staller}

\fancyfoot[L]{}
\fancyfoot[C]{\thepage /\pageref{LastPage}}
\fancyfoot[R]{}

\renewcommand{\textheight}{700px}
\renewcommand{\footskip}{10px}
\newcommand*\xor{\mathbin{\oplus}}
\begin{document}	
	\section*{Aufgabe 1}
	
	\begin{enumerate}[(a)]
		\item Der Floorplan stellt die einzelnen funktionalen Baugruppen angeordnet auf dem Chip dar. Er ist damit der Geometrie-Achse zuzuordnen.
		\item Die Netzliste ist die Beschreibung der Struktur der einzelnen Komponenten. Sie ist damit der Struktur-Achse zuzuordnen.
		\item In der Register-Transfer-Ebene.
		\item Von der algorithmischen Ebene zur RT-Ebene gelangt man durch High-Level-Synthese. Durch Einbinden von Bibliotheken kann die RTL-Beschreibung dann von der Struktur- auf die Verhaltenssicht abgebildet werden.
		\item Intellectual Properties sind Entwürfe von Subsystemen, die zu einem komplexeren System kombiniert werden können. 
		Es gibt drei Arten von IPs:
		
		\begin{enumerate}[i.]
			\item Soft IP: Die Komponente liegt in höherer Hardwarebeschreibungssprache wie VHDL oder Verilog vor und kann deshalb noch angepasst werden.
			\item Hart IP: Die Komponente liegt nicht in höherer Hardwarebeschreibungssprache wie VHDL oder Verilog vor, sondern wird als bereits fertiggestelltes Layout geliefert.
			\item Firm IP:  Die Komponente ist bereits mit strukturellen Daten zur Platzierung der Baugruppen versehen, kann jedoch noch für verschiedene Anwendungsfälle konfiguriert werden. 
		\end{enumerate}		
				
		Besonders beim Plattformbasierten Entwurf werden IPs eingesetzt, vereinzelt aber auch beim Blockbasierten Entwurf.
		\item \hfil
		
		\begin{enumerate}[i.]
			\item Nexperia HW-Architektur
			\item TI OMAP 5912 HW-Architektur
			\item Qualcom MSM5100 HW-Architektur
		\end{enumerate}
	\end{enumerate}

	
\end{document}

